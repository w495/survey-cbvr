%& -shell-escape

\documentclass[utf8x, sts, 14pt, a4paper,oneside,fleqn]{extarticle}
% fleqn --- сдвигает формулы влево

%% Варианты {}:
% book
% report
% article
% letter
% minimal (???)

\usepackage{styles/init}



% подключаем набор стилей
% там были определены базовые настройки шрифтов
% и пакетов роботы с графикой и листингами

% При не обходимости шрифты следует переопределить
% потому что, если в Вашей системе
% не окажется нужных шрифтов, pdf не соберется

% текущее положение включаемых файлов --- ./src

\hypersetup{
    pdftitle={Обзор методов комплексного ассоциативного поиска видео},
    pdfauthor={Илья w-495 Никитин},
    pdfcreator={XeLaTeX + w-495},
    pdfsubject={Обзор методов комплексного ассоциативного поиска видео},
    pdfproducer={w-495},
    pdfkeywords={\workpdfkeywords}
}


\renewcommand{\headname}[1]{%
    %\pdfdate#1
    © Ilya \href{mailto:w@w-495.ru}{w495} Nikitin~— %
    An Overview Of Complex Content-Based Video Retrieval, %
    Обзор методов комплексного ассоциативного поиска видео,
    2014.%
}

\begin{document}
    \begin{onehalfspacing}
        \subimport{src/}{cover}
        %\subimport{src/}{titlepage}
        \subimport{src/}{preface}
        \pagebreak
        % содержание
        \tableofcontents
        \pagebreak
        %\printglossary[title=Словарь терминов,toctitle=Словарь терминов]
        %\pagebreak
        % обзор
        \pagebreak
        \subimport{src/}{survey}
        \pagebreak
        % список использованных источников
        \subimport{src/}{biblio}

        \subimport{src/}{postface}
        \printindex
    \end{onehalfspacing}
\end{document}

%%% Local Variables:
%%% mode: latex
%%% TeX-PDF-mode: t
%%% End:

