% \Asection{Аннотация}

\noindent УДК 004.932.4

\begin{flushright}
\sf\small
И.\,К.~Никитин \\
Московский Авиационный Институт \\
(Национальный Исследовательский Университет), \\
Волоколамское шоссе, д.~4, \\
г.~Москва, 125993, Россия \\
Email: \href{mailto: w@w-495.ru}{w@w-495.ru}
\end{flushright}

\Atitle{Обзор методов комплексного \\ассоциативного поиска видео}

В~статье предлагается обзор различных существующих методов
ассоциативного поиска по~видео.
В~течение прошлого десятилетия наблюдался стремительный рост
количества видео размещаемых в~Интернете,
что~создало острую необходимость в~появлении поиска по~видео.
Видео имеет сложную структуру. Одна и~та~же информация
может быть выражен различными способами.
Это серьёзно усложняет задачу видео-поиска.
Заголовки и~описания видео не~могут
дать полного представления о~самом видео,
что~влечет за~собой необходимость использования
ассоциативного поиска по~видео.
Существует семантический разрыв между~низкоуровневыми
характеристиками видео и~восприятием пользователей.
Комплексный ассоциативный видео‑поиск
может рассматриваться как~связующее звено между~обычным поиском
и~смысловым поиском по~видео.

{\bf Ключевые слова:}
анализ видео;
аннотирование видео;
видео-поиск;
кадры;
классификация видео;
нечеткие дубликаты видео;
ранжирование видео;
сцены;
съёмки.





