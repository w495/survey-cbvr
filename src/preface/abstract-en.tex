
% \Asection{Abstract}

\begin{flushright}
\sf\small
Ilya Nikitin \\
postgraduate student \\
of Moscow Aviation Institute \\
(National Research University) \\
Moscow, Russia \\
E-mail: \href{mailto: w@w-495.ru}{w@w-495.ru} \\
\end{flushright}


\Atitle{An Overview Of Complex \\Content-Based Video Retrieval}



The paper focuses~on an~overview of the different existing methods
in content-based video retrieval.
During the last decade there was a~rapid growth
of video posted on the Internet.
This imposes urgent demands on video retrieval.
Video has a complex structure and can express the same idea
in different ways. This makes the task of searching for video more complicated.
Video titles and text descriptions cannot give the hole information
about objects and events in the video.
This creates a need for content-based video retrieval.
There is a semantic gap between low-level video features,
that can be extracted, and the users' perception.
Complex content-based video retrieval
can be regarded as the bridge between traditional retrieval
and semantic-based video retrieval.


{\bf Keywords:}
frames;
near-duplicates video;
scenes;
shots;
video annotation;
video classification;
video mining;
video reranking;
video retrieval.



