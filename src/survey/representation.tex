
\section{Представление видео}


В~работе \cite{Haase:95} была сформулировала проблема
машинного представления видео.
В~работе \cite{Chih-Wen:2007} разработаны многослойные,
графические аннотации видео~— мультимедийные
потоки\index{Мультимедийный поток}.
Они~представляют собой визуальный язык
как~способ представления видеоданных. Особое внимание уделено
проблеме создания глобального архива видео,
допускающего повторное использование.
Нисходящие поисковые системы используют высокоуровневые знания определенной
предметной области, чтобы генерировать надлежащие представления.

Но~как~было сказано выше, это не~самый удобный подход.
Представление, управляемое данными~— стандартный способ извлечь
низкоуровневые характеристики и~получить соответствующие представления
без~любых предварительных знаний о~предметной области.

Представления, на~основе данных могут быть сведены к~двум основным классам.
\begin{enumerate}
    \item Сигнальные признаки, которые характеризуют низкоуровневое
        аудиовизуальное содержание. К~ним можно отнести
        цветовые гистограммы, формы, текстуры,
    \item Описательное представление с~помощью текста, атрибутов или~ключевых слов.
        Авторы работы \cite{Xiangang:2011} предлагают для~описания
        видео использовать послойные графовые клики ключевых кадров
        (SKCs\index{SKC}), которые более компактны и~информативны,
        чем последовательность изображений или~ключевые кадры.
\end{enumerate}



\begin{figuredt}
    \import{vec/}{video-representation}
    \fcaption{Способы представления видео}
\end{figuredt}

