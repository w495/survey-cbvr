
\section{Анализ видео}

Интеллектуальный анализ данных в~больших базах видео стал доступен недавно.
Задачи анализа видеоинформации можно сформулировать как~выявление:
\begin{itemize}
    \item структурных закономерностей видео;
    \item закономерностей поведения движущихся объектов;
    \item характеристик сцены;
    \item шаблонов событий и~их связей;
    \item и~других смысловых атрибутов в~видео.
\end{itemize}

В~работах применяют извлечение объектов~— группировку различных экземпляров
того~же объекта, который появляется в~различных частях видео.

Для~классификации пространственных характеристик кадров применяют
метод поиска ближайших соседей\index{Поиск ближайшего соседа}\index{NNS}\
\cite{Anjulan:2009}.
%При~поиске подобных объектов выделяют стабильные связи,
%которые объединены в~значимые объектные кластеры.

Обнаружение специальных шаблонов применяется к~действиям и~событиям,
для~которых есть априорные модели, такие как~действия человека,
спортивные мероприятия, дорожные ситуации
или~образцы преступлений \cite{Quack:2006}.

Поиск моделей~— автоматическое извлечение неизвестных закономерностей в~видео.
Для~поиска моделей используют экспертные системы
с~безнадзорным\index{Обучение! Без~учителя}\
или~полуконтролируемым обучением\index{Обучение! Полуконтролируемое}.

Поиск неизвестных моделей полезен
для~изучения новых данных в~наборе видео.
Неизвестные образцы обычно находят благодаря
кластеризации различных векторов характеристик.

Для~выявления закономерностей поведения движущихся объектов
используют n-граммы\index{N-грамма}\ и~суффиксные
деревья\index{Суффиксное дерево}.
При~этом анализируют последовательности событий
по~многократным временным масштабам.

