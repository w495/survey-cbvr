
\section{Выделение признаков}

Из~полученных частей видео выделяют признаки.
К~признакам относят:
\begin{itemize}
    \item характеристики ключевых кадров;
    \item объекты;
    \item движение в~кадре;
    \item характеристики аудио и~текста.
\end{itemize}


\begin{figuredt}
    \import{vec/}{feature-classification}
    \fcaption{Классификация признаков для~выделения при~ассоциативном видео-поиске}
\end{figuredt}


\subsection{Характеристики ключевых кадров}

Выделяют цветовые, текстурные, формовые, краевые характеристики.

\subsubsection{Цвета}

Цветовые характеристики включают в~себя цветовые гистограммы,
цветовые моменты\index{Цветовой момент},
цветовые автокорреляционные коррелограммы\index{Цветовая коррелограмма},
модель гауссовых смесей (МГС)\index{МГС}\index{Модели гауссовых смесей.
При~выделении локальной цветовой информации изображения разбивают
на~блоки $~5 \times 5~$ \cite{Yan:2007}.


\subsubsection{Текстуры}

Текстурными характеристиками называют визуальные особенности поверхности
некоторого объекта. Они~не~зависят от~тона или~насыщенности цвета объекта.
Текстурные характеристики отражают однородные явления в~изображениях.
Для~выделение текстурной информации
из~видео применяют фильтры Габора\index{Фильтры Габора}\ \cite{Adcock:2004}.

\subsubsection{Контуры}

Контурные или~формовые характеристики, описывают формы объектов в~изображениях.
Они~могут быть извлечены из~контуров или~областей объектов.


\subsubsection{Края}

В работе \cite{Hauptmann:2003} был описан дескриптор гистограммы границ (EDH — Edge Direction Histogram). Изображение разбивается на равных 9 ячеек. В каждой ячейке для выделения границ применяет детектор Канни. Изображение сглаживается для удаления шума. помечаются границы в местах, где градиент приобретает максимальное значение, локальные максимумы отмечаются как потенциальные границы, некоторые из них отсекаются порогами. Итоговые границы определяются путём подавления всех границ, несвязанных с определенными (сильными) границами.
Далее полученные границы представляются в виде вектора-гистограммы. Вектор содержит 73 компоненты. 72 элемента вектора характеризуют направление границы. Направление измеряется в градусах с шагом 5 и один оставшийся элемент равен количеству пикселей попавшив в эту границу.

\subsection{Характеристики объектов}

Такие характеристики включают параметры областей изображения,
которые соответствуют объектам:
\begin{itemize}
    \item основной цвет;
    \item текстуру;
    \item размер и~т.~д.
\end{itemize}

В~работе \cite{Sivic:2005} предложена система поиска лиц.
По~видео-запросу с~конкретным человеком
система способна выдать ранжированный список съёмок с~этим человеком.
Текстовая индексация и~поиск приводят к~расширению семантики запроса
и~делают возможным использования Glimpse-метода\index{Glimpse}\ (agrep\index{Agrep}),
для~поиска нечеткого соответствия \cite{Li:2002}.


\subsection{Характеристики движения}

Характеристики движения ближе к~смысловым понятиям,
чем характеристики ключевых статических кадров и~объектов.
Движение в~видео может быть вызвано движением камеры
и~движением предметов в~кадре.

Движения камеры такие как~«приближение или~удаление»,
«панорамирование влево или~вправо» и~«смещение вверх или~вниз»
используются для~индексации видео.
Движения объектов на~данный момент являются предметном исследований.

\subsection{Звуковые характеристики}

Преимущество аудио-подходов состоит в~том,
что~они~обычно требуют меньше вычислительных ресурсов, чем визуальные методы.
Кроме~того, аудио-записи могут быть очень короткими.

Многие звуковые характеристики выбраны на~основе человеческого восприятие звука.
Характеристики аудио можно разделить на~три уровня \cite{Chen:2008}:
\begin{itemize}
    \item низкоуровневая акустика, такая как~средняя частота для~кадра;
    \item средний уровень, как~признак объекта, например звук скачущего мяча;
    \item высокоуровневые, такие как~речь и~фоновая музыка;
        играющая в~определенных типах видео.
\end{itemize}


В~работе \cite{Seyerlehner:2010} используют блочные характеристики аудио.
Аудио-поток при~этом разделяется на~отрезки в~2048 отсчетов.
Для~выделения таких характеристик применяют функцию Ханна\index{Функция Ханна}\
и~логарифмическую шкалу.















