

\section{Классификация видео}

Задача классификации состоит в~том,
чтобы отнести видео к~предопределенной категории.
Для~этого используют характеристики видео или
результаты интеллектуального анализа данных.

Классификация видео~— хороший способ увеличить
эффективность видео-поиска.
Семантический разрыв между~низкоуровневыми данными
и~интерпретацией наблюдателя, делает ассоциативную классификацию
очень трудной задачей.

Смысловая классификация видео может быть выполнена
на~трех уровнях \cite{Tamizharasan:2013}:
\begin{itemize}
    \item жанры:
    \begin{itemize}
        \item например, «фильмы», «новости»,
                «спортивные соревнования», «мультфильмы», «реклама» и~т.~д.
    \end{itemize}
    \item события видео;
    \item и~объекты в~видео.
\end{itemize}


\begin{figuredt}
    \import{vec/}{video-levels}
    \fcaption{Уровни классификации видео}
\end{figuredt}



\subsection{Жанры}

Жанровая классификация разделяет видео на~подмножество соответствующее жанру
и~несоответствующее \cite{Wu:2012}.

В~работе \cite{Jiang:2007} предложена классификация большого числа
видео только по~заголовку видео.
Для~этого использован поэтапный метод опорных
векторов\index{Метод опорных векторов}.

Видео классифицируют также на~основе статистических моделей различных жанров.
Для~этого анализируют структурные свойства:
статистику цвета, съёмки, движение камеры и~объектов.
Свойства используются, чтобы получить более абстрактные атрибуты стиля.
К~абстрактным атрибутам стиля можно отнести:
панорамирование камеры и~изменение масштаба, речь и~музыку.
Строят отображение этих атрибутов на~жанры видео.

В~работе \cite{Ionescu:2012} для~классификации жанров используется
комбинация из~четырех дескрипторов:
\begin{itemize}
    \item блоковый аудио дескриптор:
    \begin{itemize}
        \item захватывает локальную временну́ю информацию;
    \end{itemize}
    \item дескриптор визуальной временно́й структуры:
    \begin{itemize}
        \item использует информацию о~смене съёмок,
        \item оценивает количество съёмок
            за~определенный интервал времени («ритм» видео),
        \item описывает «активные» и~«не~активные» смены съемок;
    \end{itemize}
    \item дескриптор цвета:
    \begin{itemize}
        \item использует статистику распределения цвета,
        элементарных оттенков, цветовых свойств, и~отношений между~цветами;
    \end{itemize}
    \item статистика фигур контуров.
\end{itemize}

Были проведены эксперименты на~видеоматериалах общей продолжительностью
91 час видео. Классификация проводилась на~семи жанрах видео:
мультфильмы, реклама, документальные фильмы, художественные фильмы,
музыкальные клипы, спортивные соревнования и~новости.
Комплексный дескриптор позволил авторам достичь
точности  87﹪~— 100﹪ и~полноты 77﹪~— 100﹪.



\subsection{События}

Событие может быть определено как~любое явление в~видео,
которое
\begin{itemize}
    \item может быть воспринято зрителем;
    \item играет роль для~представления содержимого.
\end{itemize}
Каждое видео может состоять из~многих событий,
и~каждое событие может состоять из~многих подсобытий.
Таким образом складывается иерархическая модель \cite{Chang:2002}.

\subsection{Объекты}

Объектная классификация является самым низкоуровневым типом классификации.
Съёмки классифицируют тоже на~основе объектов.
Объекты в~съёмках представлены с~помощью параметров цвета, текстуры и~траектории.
В~работе \cite{Hong:2005} для~кластеризации связанных съёмок
используется нейронная сеть.
Каждый кластер отображен на~одну из~12 категорий.
Объекты разделяются по~положению в~кадре и~характеру движения.


\section{Аннотирование видео}

Процесс присваивания переопределенных смысловых понятий фрагментам видео
называют аннотированием. Примеры смысловых понятий: человек,
автомобиль, небо и~гуляющие люди.

Аннотирование видео подобно классификации, за~исключением двух различий.
\begin{enumerate}
    \item Для~классификаций важны жанры, а~для~аннотирования понятия.
        Жанры и~понятия имеют различную природу, несмотря на~то,
        что~некоторые методы могут быть использованы в~обоих задачах.
    \item Классификация видео применяется к~полным видео,
        в~то время как~аннотируют обычно фрагменты \cite{Yang:2007}.
\end{enumerate}

Аннотирование, основанное на~обучении, необходимо для~анализа
и~понимания видео. Было предложено много различных способов
автоматизации процесса.

Например, в~работе \cite{Zhang:2012} было
разработано «быстрое
полуконтролируемое\index{Обучение! Полуконтролируемое}\ графовое
обучение на~нескольких экземплярах»
(Fast Graphbased Semi-Supervised Multiple Instance Learning~— FGSSMIL\index{FGSSMIL}).
Алгоритм работает в~рамках общей платформы для~разных типов видео одновременно
(спортивные передачи, новости, художественные фильмы).
Для~обучения модели используется небольшое число видео, размеченных вручную,
и~значительный объем не~размеченного материала.

В~работе \cite{Weal:2012} предлагается создавать
частичную ручную аннотацию видео
как~часть практической профессиональной подготовки.
Авторы рассматривают лабораторные занятия студентов-медиков.
Во~время занятия идет запись видео.
Кроме~того одновременно происходит запись изменения состояния
тренировочного манекена (виртуального пациента).
Таким образом, к~записанному видео добавляется семантическая разметка
на~основе показаний датчиков манекена.
После происходит разбор занятия и~анализ допущенных ошибок,
В~результате к~видео добавляется разметка, созданная самими студентами.













