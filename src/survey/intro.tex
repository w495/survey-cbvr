
%
% За~последние несколько лет объем носителей мультимедийной информации
% вырос в~несколько раз. Одновременно с~этим уменьшилась стоимость хранения.
% Стало возможным публичное размещение большого числа видео-материалов.
% В~сложившейся ситуации необходим простой
% и~гибкий поиск неструктурированным мультимедийных данных.
% В~статье предлагается обзор различных существующих методов
% ассоциативного поиска по~видео.
%
%

\section{Введение}

% В~связи с~увеличением пропускной способности сетей,
% многие пользователи получили доступ к~видео в~Интернете.
% Для~примера, каждую минуту на~сайт YouTube\index{YouTube}\  загружаются
% более 48 часов новых видео.
% Более 14 миллиардов клипов были просмотрены в~мае 2010.

\noindent
Видео обладает тремя отличиями от~других типов медийных
ресурсов \cite{Nabeel:2014}.
\begin{enumerate}
    \item Видео более насыщено информацией, чем, например изображения.
    \item Видео содержит много необработанных данных.
    \item Видео имеет слабо выраженную структуру.
\end{enumerate}
Эти особенности делают поиск и~индексацию видео весьма сложными.
Ранее, видео базы были крошечными.
Поиск и~индексация в~таких базах были основаны
на~ручной аннотации по~ключевым словам.
Недавно, размеры таких баз~сильно возросли
и~возникла необходимость автоматического анализа видео
с~минимальным привлечением человеческого труда.

В~длинном видео сложно автоматизировано найти интересующий отрывок.
А, размечать и~искать видео вручную весьма трудоемко.
Смысловой разрыв между~низкоуровневой информацией
и~потребностями пользователя, заставляет работать
с~видео на~более высоком уровне.
Тем не~менее, большинство методов поиска следуют парадигме
прямого отображения низкоуровневых характеристик видео
на~смысловые понятия.
Этот подход требует предварительной обработки данных.
А~результаты такого отображения не~будут устойчивы.
Без~учета конкретной предметной области задача кажется неразрешимой.

Анализ большого объема видео-данных для~выделения нужной
информации является сложной задачей.
Для~ее решения применяют ассоциативный поиск.
В~англоязычной литературе ассоциативный видео-поиск называют
«content based video retrieval» (CBVR\index{CBVR})~— 
поиск по~содержимому.

\subsection{Конкретные применения}

Существует много разных применений видео-поиска. Например:
\begin{itemize}
    \item исследование электронной коммерции
        и~анализ тенденций выбора и~упорядочивания материалов;
    \item быстрый поиск видео в~массиве данных,
    \item анализ новостных событий \cite{Peng:2005},
    \item интеллектуальное администрирование и~контроль интернет-видео~—
        поиск нужных видео и~отсеивание нежелательных.
\end{itemize}

Последнее время стало появляться много клипов
с~очень схожим содержанием (нечеткие
дубликаты\index{Нечеткие дубликаты}\ видео).
Задача эффективной идентификации нечетких дубликатов играет ключевую
роль в~задачах поиска, защите авторских прав, и~многих других.

Ассоциативный поиск используется для~автоматического
реферирования видео, анализа новостных событий,
видеонаблюдения, и~в~образовательных целях \cite{Dimitrova:2002}.

Эти и~многие другие прикладные задачи мотивируют ученых заниматься
проблемами индексации и~поиском по~содержимому.

\subsection{Исследования}

Существует два вида исследований: научные и~промышленные.

Примером крупных научных исследований может служить
регулярная конференция TRECVid\index{TRECvid}~(Text REtrieval Conference: Video),
которую с~2003 года проводит Американский Национальный институт
стандартов и~технологий~(NIST\index{NIST}).
Многие участники присылают свои алгоритмы для~выпуска коллективной монографии,
например \cite{Over:2011}, \cite{Awad:2014} и \cite{Smeaton:2010}.

% Американский Национальный институт стандартов
% и~технологий~(NIST\index{NIST}) с~2003 года проводит
% конференцию TRECvid\index{TRECvid}~(Text REtrieval Conference Video).
% Эта конференция стимулирует работы в~области поиска и~анализа видео.
% TRECVid обеспечивает большое число экспериментов с~видео.
% Многие участники присылают свои алгоритмы для~выпуска коллективной монографии,
% например \cite{Over:2011}, \cite{Awad:2014} и \cite{Smeaton:2010}.

В~промышленных исследованиях важную роль играют экспертные группы
стандарта TV-Anytime\index{TV-Anytime}\ и~MPEG\index{MPEG}~(Moving
Picture Experts Group) \cite{Pereira:2008}.
Существуют работы, в~которых видео-поиск рассматривается
в~рамках задачи сжатия видео \cite{Babu:2007}.


\subsection{Комплексность}

Видео содержит в~себе несколько видов данных.
Авторы \cite{Chung:2007:PAU} и~\cite{smeaton:2006} выделяют четыре вида.
\begin{enumerate}
    \item   Метаданные~— заголовок, автор и~описание.
    \item   Звуковая дорожка.
    \item   Тексты полученные при~помощи технологии
            оптического\index{Оптическое распознавание текста}\
            распознавания символов\index{Распознавание! текста}\
            (OCR\index{OCR}) или~при~помощи технологий
            распознавания речи\index{Распознавание! речи}.
    \item   Визуальная информация кадров видео.
\end{enumerate}


\begin{figuredt}
    \import{vec/}{video-structure}
    \fcaption{Структура видео}
\end{figuredt}

Таким образом, видео обладает комплексностью.
Комплексность (системность, мультимодальность)~— способность взаимодействовать
с~пользователем по~различным каналам информации
и~извлекать и~передавать смысл автоматически \cite{Nigay:1993}.
Комплексность видео состоит в~возможности автора выражать мысли,
используя по~крайней мере два информационных канала.
Каналы могут быть визуальными, звуковыми или~текстовыми.


\subsection{Обзоры}

Поиску и~индексации видео было посвящено много обзоров.
Обычно обзоры фокусируются на~какой-то подзадаче поиска.
Например, в~работе Смитона \cite{Smeaton:2010} дается хороший обзор
методов обнаружения границ съёмок за~семь лет работы TRECVid.
В~обзоре Сноека \cite{Snoek:2009} приведен
полный анализ смыслового поиска по~видео.
Основной акцент сделан на~оценке алгоритмов,
использующих базы данных TRECVid,
поиске видео по~семантическим понятиям.
Шеффман c соавторами \cite{Schoeffmann:2010} сделали
обзор по~интерфейсам систем видео просмотра и~их применению.
В~работе \cite{Ren:2009} приведен анализ методов
выделения пространственно-временной смысловой информации из~видео.
В~работе \cite{Zhang:2012} дают хороший обзор аннотации видео.
В~работе \cite{XinmieTian:2011} описываются свежие исследования
методов ранжирования видео.



\subsection{Общая схема}

Ассоциативный поиск видео состоит из~следующих шагов.
\begin{enumerate}
    \item Анализ временной структуры видео~—
        деление видео на~фрагменты, которое включает обнаружение границ съёмок.
    \item Определение характеристик фрагментов:
        параметров ключевых кадров,
        характеристик движения и~объектов.
    \item Извлечение информации из~характеристик.
    \item Аннотация видео, построение семантического индекса.
    \item Обработка пользовательского запроса и~выдача результата.
    \item Обратная связь и~переранжирование результатов для~улучшения поиска характеристик.
\end{enumerate}

\pagebreak


\begin{dtable}{Некоторые научные работы по~ассоциативному поиску видео}

    \renewcommand{\arraystretch}{1.4}

    \begin{tabular}{|c|c|c|}
        \hline
            Год
            & Работа
            & Тема
        \\
        \hline
        \hline
            2008 & \cite{Haase:95} & \\
            2010 & \cite{Smeaton:2010} & \\
            2011 & \cite{Kumar:2011}
            & \multirow{-3}{*}{Сегментация видео}
        \\
        \hline
            2010 & \cite{Fu:2010} & \\
            2012 & \cite{Wang:2012}
            & \multirow{-2}{*}{Автоматическое реферирование видео}
        \\
        \hline
            2012 & \cite{Chen:2012}, \cite{Zha:2012}, \cite{Wu:2012} & \\
            2013 & \cite{Paul:2013} & \\
            2014 & \cite{Nabeel:2014}
            & \multirow{-3}{*}{Индексация видео}
        \\
        \hline
            2012 & \cite{Huurnink:2012} & \\
            2013 & \cite{Tamizharasan:2013}
            & \multirow{-2}{*}{Комплексный ассоциативный поиск}
        \\
        \hline
            2011
            & \cite{Karpenko:2011}, \cite{Xiangang:2011}
            & Представление видео
        \\
        \hline
            2009 & \cite{Snoek:2009} & \\
            2012 & \cite{Jiang:2012}, \cite{Yu:2012}, \cite{Andre:2012}
            & \multirow{-2}{*}{Смысловой ассоциативный поиск}
        \\
        \hline
            2012
            & \cite{Zhang:2012}, \cite{Yu:2012}
            & Аннотирование видео
        \\
        \hline
            2012
            & \cite{Wei-Ta:2012}
            & Видео-поиск по~движению
        \\
        \hline
            2011 & \cite{XinmieTian:2011} & \\
            2012 & \cite{Zhang:2012}
            & \multirow{-2}{*}{Ранжирование видео}
        \\
        \hline
            2010 & \cite{Tahayna:2010} & \\
            2011 & \cite{Sargin:2011} & \\
            2012 & \cite{JaeDeok:2012}, \cite{Ionescu:2012}
            & \multirow{-3}{*}{Классификация видео}
        \\
        \hline
    \end{tabular}
\end{dtable}


\pagebreak


\begin{figuredt}
    \import{vec/}{video-search-sheme}
    \fcaption{Схема поиска по~видео}
\end{figuredt}














