
%
% За последние несколько лет объем носителей мультимедийной информации
% вырос в несколько раз. Одновременно с~этим уменьшилась стоимость хранения.
% Стало возможным публичное размещение большого числа видео-материалов.
% В~сложившейся ситуации необходим простой
% и~гибкий поиск неструктурированным мультимедийных данных.
% В статье предлагается обзор различных существующих методов
% ассоциативного поиска по видео.
%
%

\section{Введение}

В~связи с~увеличением пропускной способности сетей,
многие пользователи получили доступ к~видео в~Интернете.
Для примера, каждую минуту на~сайт YouTube\index{YouTube}\  загружаются
более 48 часов новых видео.
Более 14 миллиардов клипов были просмотрены в~мае 2010.

В~длинном видео сложно автоматизировано найти интересующий отрывок.
А, размечать и~искать видео вручную весьма трудоемко.
Смысловой разрыв между низкоуровневой информацией
и~потребностями пользователя, заставляет работать
с~видео на~более высоком уровне.
Тем не~менее, большинство методов поиска следуют парадигме
прямого отображения низкоуровневых характеристик видео
на~смысловые понятия.
Этот подход требует предварительной обработки данных.
А результаты такого отображения не~будут устойчивы.
Без учета конкретной предметной области задача кажется неразрешимой.
Последнее время стало появляться много клипов
с~очень схожим содержанием (нечеткие дубликаты\index{Нечеткие дубликаты} видео).

Задача эффективной идентификации нечетких дубликатов играет ключевую
роль в~задачах поиска, защите авторских прав, и~многих других.
Анализа большого объема видео-данных для~выделения нужной
информации является сложной задачей.
Для ее решения применяют ассоциативный поиск.
В англоязычной литературе ассоциативный видео-поиск называют
«content based video retrieval» (CBVR\index{CBVR}) —
поиск по содержимому.

Ассоциативный поиск используется для~автоматического
реферирования видео, анализа новостных событий,
видеонаблюдения, и~в образовательных целях \cite{Dimitrova:2002}.

Видео содержит в~себе несколько видов данных.
Авторы \cite{Chung:2007:PAU} и~\cite{smeaton:2006} выделяют четыре вида.
\begin{enumerate}
    \item Метаданные~— заголовок, автор и~описание;
    \item Звуковая дорожка;
    \item Тексты полученные при~помощи технологии
          оптического\index{Оптическое распознавание текста}\
          распознавания символов\index{Распознавание текста}\
          (OCR\index{OCR});
    \item Визуальная информация кадров видео.
\end{enumerate}


\begin{figuredt}
    \import{vec/}{video-structure}
    \fcaption{Структура видео}
\end{figuredt}

Таким образом, видео обладает комплексностью.
Комплексность (системность, мультимодальность)~— способность взаимодействовать
с~пользователем по различным каналам информации
и~извлекать и~передавать смысл автоматически \cite{Nigay:1993}.

Комплексность видео состоит в~возможности автора выражать мысли,
используя по крайней мере два информационных канала.
Каналы могут быть визуальными, звуковыми или текстовыми.


\begin{dtable}{Некоторые научные работы по ассоциативному поиску видео}

    \renewcommand{\arraystretch}{1.4}

    \begin{tabular}{|c|c|c|}
        \hline
            Год
            & Работа
            & Тема
        \\
        \hline
        \hline
            2008 & \cite{Haase:95} & \\
            2011 & \cite{Kumar:2011}
            & \multirow{-2}{*}{Сегментация видео}
        \\
        \hline
            2010 & \cite{Fu:2010} & \\
            2012 & \cite{Wang:2012}
            & \multirow{-2}{*}{Автоматическое реферирование видео}
        \\
        \hline
            2012 & \cite{Chen:2012}, \cite{Zha:2012}, \cite{Wu:2012} & \\
            2013 & \cite{Paul:2013} & \\
            2014 & \cite{Nabeel:2014}
            & \multirow{-3}{*}{Индексация видео}
        \\
        \hline
            2012 & \cite{Huurnink:2012} & \\
            2013 & \cite{Tamizharasan:2013}
            & \multirow{-2}{*}{Комплексный ассоциативный поиск}
        \\
        \hline
            2011
            & \cite{Karpenko:2011}, \cite{Xiangang:2011}
            & Представление видео
        \\
        \hline
            2012
            & \cite{Jiang:2012}, \cite{Yu:2012}, \cite{Andre:2012}
            & Смысловой ассоциативный поиск
        \\
        \hline
            2012
            & \cite{Zhang:2012}, \cite{Yu:2012}
            & Аннотирование видео
        \\
        \hline
            2012
            & \cite{Wei-Ta:2012}
            & Видео-поиск по движению
        \\
        \hline
            2011 & \cite{XinmieTian:2011} & \\
            2012 & \cite{Zhang:2012}
            & \multirow{-2}{*}{Ранжирование видео}
        \\
        \hline
            2010 & \cite{Tahayna:2010} & \\
            2011 & \cite{Sargin:2011} & \\
            2012 & \cite{JaeDeok:2012}, \cite{Ionescu:2012}
            & \multirow{-3}{*}{Классификация видео}
        \\
        \hline
    \end{tabular}
\end{dtable}

В~работе \cite{Zhang:2012} дают хороший обзор аннотации видео.
В~работе \cite{XinmieTian:2011} описываются свежие исследования
методов ранжирования видео.

Ассоциативный поиск видео состоит из~следующих шагов.
\begin{enumerate}
    \item Анализ временной структуры видео~—
        деление видео на~фрагменты, которое включает обнаружение границ съёмок.
    \item Определение характеристик фрагментов.
    \item Извлечение информации из~характеристик.
    \item Аннотация видео, построение семантического индекса.
    \item Обработка пользовательского запроса и~выдача результата.
    \item Обратная связь и~переранжирование результатов для~улучшения поиска характеристик.
\end{enumerate}


\begin{figuredt}
    \import{vec/}{video-search-sheme}
    \fcaption{Схема поиска по~видео}
\end{figuredt}


