
\section{Заключение}

Многие вопросы остаются открытыми и~требуют дальнейшего исследования,
особенно в~следующих областях.

Большинство современных подходов индексации видео
сильно зависят от~предварительных знаний о предметной области.
Это ограничивает их~расширяемость для~новых областей.
Устранение зависимости от~предварительных знаний~—
важная задача будущих исследований.

Индексация и~поиск видео в~среде «облачных» вычислений сформировали
новое направление исследований видео-поиска.
Важной особенностью «облачных» вычислений является то,
что искомые видео и~сама база данных меняются динамически.

Современные подходы к~смысловому поиску видео,
как правило, используют набор текстов для~описания визуального содержания видео.
В~этой области пока осталось много неразрешенных вопросов.
Например, отдельной темой для~исследования может быть
эмоциональная семантика видео \cite{Tamizharasan:2013}.
Эмоциональная семантика описывает
человеческие психологические ощущения,
такие как радость, гнев, страх, печаль, и~пр.

Эмоциональный видео-поиск~— поиск материалов,
которые вызывают конкретные чувства у~зрителя.
Для имитации человеческого восприятия могут быть использованы
новые подходы к~видео-поиску.

Темой для~дальнейшего изучения является
мультимедийный человеко-машинный интерфейс,
в~частности:
\begin{itemize}
    \item расположение мультимедийной информации;
    \item удобство интерфейса для~решения задач пользователя;
    \item пригодность интерфейса для~оценки и~обратной связи пользователей;
    \item и~способность интерфейса адаптироваться к~привычкам запроса пользователей
        и~отражать их индивидуальность.
\end{itemize}

Организация и~визуализация результатов поиска~— также
интересная тема исследования. На данный момент проблема
сочетания множественных информационных моделей
на~различных уровнях абстракции остается неразрешенной.

Эффективное использование информации о~движении имеет большое значение
для~поиска видео. Важными задачами направления являются:
\begin{itemize}
    \item способность различать движения фона и~переднего плана;
    \item обнаружение движущиеся объектов и~определять события в~кадре;
    \item объединение статических характеристик и~характеристик движения;
    \item построение индекса движения.
\end{itemize}

Интересными вопросами для~исследования остаются:
\begin{itemize}
    \item быстрый видео-поиск с~помощью иерархических индексов;
    \item адаптивное обновление иерархической индексной модели;
    \item обработка временных характеристик видео во~время создания и~обновления индекса;
    \item динамические меры сходства видео на~основе выбора статистических функций.
\end{itemize}

