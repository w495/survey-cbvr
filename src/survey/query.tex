
\section{Обработка запроса}

После построения поискового индекса может быть выполнен ассоциативный поиск.
Поисковая выдача оптимизируются на~основе связи между запросами.

\subsection{Типы запросов}


Существует две категории запросов:
семантические и~не семантические.


\begin{figuredt}
    \import{vec/}{video-query-types}
    \fcaption{Типы запросов при ассоциативном видео-поиске}
\end{figuredt}


\subsubsection{Семантические запросы}


К~семантическим запросам относят наборы ключевых слов
и~поисковые фразы.
Ключевые слова~— наиболее очевидный и~простой вид запроса.
При~таких запросах частично учитывается семантика видео.
Поисковые фразы или запросы на~естественном языке~—
самый естественный и~удобный способ взаимодействия
человека с~поисковой системой. Для выбора и~ранжирования видео
используется смысловая близость слов \cite{Aytar:2008}.

\subsubsection{Не семантические запросы}

Не~семантические запросы используются для~поиска по образцу,
эскизам, объектам и~т.д..
Запросом может быть изображение или видео.

\paragraph{Поиск по образцу}

При~таком поиске из~запроса выделяют низкоуровневые характеристики
и~сравнивают их с~данными в~базе с~помощью меры сходства.

\paragraph{Поиск по эскизу}

Пользователи могут изобразить нужное видео с~помощью эскиза.
Далее для~эскиза применяется поиск по образцу.

\paragraph{Поиск объекта}

В~качестве запроса выступает изображение объекта.
Система находит и~возвращает все вхождения объекта
в~материалах из~базы \cite{Sivic:2006}.
В~отличие от~предыдущих видов запросов,
в~данном случае, привязка происходит не~к видео,
а~именно по изображенному объекту.


\subsection{Оценка сходства}


Критерии близости видео является важным фактором при~поиске.
Выделяют несколько способов сравнения видео:
\begin{itemize}
    \item сравнение характеристик;
    \item сравнение текста;
    \item сравнение онтологий.
\end{itemize}

Применяют также комбинации методов.
Выбор конкретного метода зависит от~типа запроса.

\subsection{Сравнение характеристик}

При~сравнение характеристик видео оценивают среднее расстояние между
особенностями соответствующих кадров \cite{Browne:2005}.

\subsection{Сравнение текста}

Для~сравнения запроса и описания видео
применяют текстовое сопоставление.
Описание и~запрос нормализуют,
а~затем вычисляют их смысловое сходство,
используя пространственные векторные модели \cite{Snoek:2007}.

\subsection{Сравнение онтологий}

При~сравнении онтологии оценивают смысловое сходство
отношений между ключевыми словами запроса
и~описанием аннотированного видео \cite{Aytar:2008}.

Для усиления влияния смысловых понятий
автоматически подбирают комбинации методов.
Для этого исследуют различные стратегии на~учебном наборе видео.

\subsection{Оценка релевантности}

Видео из~поисковой выдачи оцениваются или пользователем или автоматически.
Эту оценку используют для~уточнения дальнейших поисков.
Обратная связь релевантности устраняет разрыв между смысловым понятием
адекватности поискового ответа и~низкоуровневым представлением видео.

Явная обратная связь\index{Обратная связь!Явная}\ предлагает пользователю выбрать
релевантные видеоролики из~ранее полученных ответов.
На основе мнений пользователей в~системы меняют коэффициенты
мер подобия \cite{Chen:2008:a}.

Неявная обратная связь\index{Обратная связь!Неявная}\ уточняет результаты поиска
на~основе кликов и~переходов пользователя.

Псевдообратная связь\index{Обратная связь!Псевдообратная}\
выделяет положительные и~отрицательные выборки
из~предыдущих результатов поиска без участия человека.

Рассматривая текстовую и~визуальную информацию
с~вероятностной точки зрения, визуальное ранжирование
можно сформулировать как задачу байесовской оптимизации\index{Байесовская оптимизация}.
Такое прием называют байесовским визуальным ранжированием.

