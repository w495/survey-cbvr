
\noindent УДК 004.932.4

\Asection{Аннотация}

\Atitle{Обзор методов комплексного \\ассоциативного поиска видео}

{\small

В статье предлагается обзор различных существующих методов
ассоциативного поиска по видео.
В течение прошлого десятилетия наблюдался стремительный рост
количества видео размещаемых в Интернете,
что создало острую необходимость в появлении поиска по видео.
Видео имеет сложную структуру. Одна и та же информация
может быть выражен различными способами.
Это серьёзно усложняет задачу видео-поиска.
Заголовки и описания видео не могут
дать полного представления о самом видео,
что влечет за собой необходимость использования
ассоциативного поиска по видео.
Существует семантический разрыв между низкоуровневыми
характеристиками видео и восприятием пользователей.
Комплексный ассоциативный видео-поиск
может рассматриваться как связующее звено между обычным поиском
и смысловым поиском по видео.

{\bf Ключевые слова:}
анализ видео;
аннотирование видео;
видео-поиск;
кадры;
классификация видео;
нечеткие дубликаты видео;
ранжирование видео;
сцены;
съёмки.

}

\Asection{Abstract}

\Atitle{An Overview Of Complex \\Content-Based Video Retrieval Methods}

{\small

The paper focuses on an overview of the different existing methods
in content-based video retrieval.
During the last decade there was a~rapid growth
of video posted on the Internet.
This imposes urgent demands on video retrieval.
Video has a complex structure and can express the same idea
in different ways. This makes the task of searching for video more complicated.
Video titles and text descriptions cannot give the hole information
about objects and events in the video.
This creates a need for content-based video retrieval.
There is a semantic gap between low-level video features,
that can be extracted, and the users' perception.
Complex content-based video retrieval
can be regarded as the bridge between traditional retrieval
and semantic-based video retrieval.


{\bf Keywords:}
frames;
near-duplicates video;
scenes;
shots;
video annotation;
video classification;
video mining;
video reranking;
video retrieval.

}




%
% За последние несколько лет объем носителей мультимедийной информации
% вырос в несколько раз. Одновременно с~этим уменьшилась стоимость хранения.
% Стало возможным публичное размещение большого числа видео-материалов.
% В~сложившейся ситуации необходим простой
% и~гибкий поиск неструктурированным мультимедийных данных.
% В статье предлагается обзор различных существующих методов
% ассоциативного поиска по видео.
%
%

\pagebreak


\section{Введение}

В~связи с~увеличением пропускной способности сетей,
многие пользователи получили доступ к~видео в~Интернете.
Для примера, каждую минуту на~сайт YouTube\index{YouTube}\  загружаются
более 48 часов новых видео.
Более 14 миллиардов клипов были просмотрены в~мае 2010.

В~длинном видео сложно автоматизировано найти интересующий отрывок.
А, размечать и~искать видео вручную весьма трудоемко.
Смысловой разрыв между низкоуровневой информацией
и~потребностями пользователя, заставляет работать
с~видео на~более высоком уровне.
Тем не~менее, большинство методов поиска следуют парадигме
прямого отображения низкоуровневых характеристик видео
на~смысловые понятия.
Этот подход требует предварительной обработки данных.
А результаты такого отображения не~будут устойчивы.
Без учета конкретной предметной области задача кажется неразрешимой.
Последнее время стало появляться много клипов
с~очень схожим содержанием (нечеткие дубликаты\index{Нечеткие дубликаты} видео).

Задача эффективной идентификации нечетких дубликатов играет ключевую
роль в~задачах поиска, защите авторских прав, и~многих других.
Анализа большого объема видео-данных для~выделения нужной
информации является сложной задачей.
Для ее решения применяют ассоциативный поиск.
В англоязычной литературе ассоциативный видео-поиск называют
«content based video retrieval» (CBVR\index{CBVR}) —
поиск по содержимому.

Ассоциативный поиск используется для~автоматического
реферирования видео, анализа новостных событий,
видеонаблюдения, и~в образовательных целях \cite{Dimitrova:2002}.

Видео содержит в~себе несколько видов данных.
Авторы \cite{Chung:2007:PAU} и~\cite{smeaton:2006} выделяют четыре вида.
\begin{enumerate}
    \item Метаданные~— заголовок, автор и~описание;
    \item Звуковая дорожка;
    \item Тексты полученные при~помощи технологии
          оптического\index{Оптическое распознавание текста}\
          распознавания символов\index{Распознавание текста}\
          (OCR\index{OCR});
    \item Визуальная информация кадров видео.
\end{enumerate}


\begin{figuredt}
    \import{tikz/}{video-structure}
    \fcaption{Структура видео}
\end{figuredt}

Таким образом, видео обладает комплексностью.
Комплексность (системность, мультимодальность)~— способность взаимодействовать
с~пользователем по различным каналам информации
и~извлекать и~передавать смысл автоматически \cite{Nigay:1993}.

Комплексность видео состоит в~возможности автора выражать мысли,
используя по крайней мере два информационных канала.
Каналы могут быть визуальными, звуковыми или текстовыми.


\begin{dtable}{Некоторые научные работы по ассоциативному поиску видео}

    \renewcommand{\arraystretch}{1.4}

    \begin{tabulary}{\textwidth}{|R|c|L|}
        \hline
            Год
            & Работа
            & Тема
        \\
        \hline
        \hline
            2008 & \cite{Haase:95} & \\
            2011 & \cite{Kumar:2011}
            & \multirow{-2}{*}{Сегментация видео}
        \\
        \hline
            2010 & \cite{Fu:2010} & \\
            2012 & \cite{Wang:2012}
            & \multirow{-2}{*}{Автоматическое реферирование видео}
        \\
        \hline
            2012 & \cite{Chen:2012}, \cite{Zha:2012}, \cite{Wu:2012} & \\
            2014 & \cite{Nabeel:2014}
            & \multirow{-2}{*}{Индексация видео}
        \\
        \hline
            2012 & \cite{Huurnink:2012} & \\
            2013 & \cite{Tamizharasan:2013}
            & \multirow{-2}{*}{Комплексный ассоциативный поиск}
        \\
        \hline
            2011
            & \cite{Karpenko:2011}, \cite{Xiangang:2011}
            & Представление видео
        \\
        \hline
            2012
            & \cite{Jiang:2012}, \cite{Yu:2012}, \cite{Andre:2012}
            & Смысловой ассоциативный поиск
        \\
        \hline
            2012
            & \cite{Zhang:2012}, \cite{Yu:2012}
            & Аннотирование видео
        \\
        \hline
            2012
            & \cite{Wei-Ta:2012}
            & Видео-поиск по движению
        \\
        \hline
            2011 & \cite{XinmieTian:2011} & \\
            2012 & \cite{Zhang:2012}
            & \multirow{-2}{*}{Ранжирование видео}
        \\
        \hline
            2010 & \cite{Tahayna:2010} & \\
            2011 & \cite{Sargin:2011} & \\
            2012 & \cite{JaeDeok:2012}, \cite{Ionescu:2012}
            & \multirow{-3}{*}{Классификация видео}
        \\
        \hline
    \end{tabulary}
\end{dtable}

В~работе \cite{Zhang:2012} дают хороший обзор аннотации видео.
В~работе \cite{XinmieTian:2011} описываются свежие исследования
методов ранжирования видео.

Ассоциативный поиск видео состоит из~следующих шагов.
\begin{enumerate}
    \item Анализ временной структуры видео~—
        деление видео на~фрагменты, которое включает обнаружение границ съёмок.
    \item Определение характеристик фрагментов.
    \item Извлечение информации из~характеристик.
    \item Аннотация видео, построение семантического индекса.
    \item Обработка пользовательского запроса и~выдача результата.
    \item Обратная связь и~переранжирование результатов для~улучшения поиска характеристик.
\end{enumerate}


\begin{figuredt}
    \import{tikz/}{video-search-sheme}
    \fcaption{Схема поиска по~видео}
\end{figuredt}



\section{Деление видео}

Деление видео включает в~себя обнаружение границ съёмок,
извлечение ключевых кадров, сегментацию сцен и~аудио.



\begin{figuredt}
    \import{tikz/}{scene-shot-frame}
    \fcaption{
        Схема визуальной сегментации видео на примере движения точек.\\
        Временная ось отложена по горизонтали.
    }
\end{figuredt}



\subsection{Обнаружение границ съёмок}

Видео делят на~фрагменты по~времени.
В~качестве таких фрагментов могут выступать съёмки.
Съёмка\index{Съёмка} (кинематографический кадр\index{Кинематографический кадр},
монтажный план\index{Монтажный план})~— отрезок киноплёнки,
на~котором запечатлено непрерывное действие между пуском
и~остановкой камеры, или между двумя монтажными склейками.

С~точки зрения семантики, самым мелким элементом видео является кадр
(фотографический кадр\index{Фотографический кадр}, кадрик\index{Кадрик}).
Съёмка является более крупным делением.
Из съёмок складываются сцены, а~из~сцен видео целиком.

Границы съёмок бывают трех типов:
\begin{itemize}
    \item линейная склейка\index{Склейка}~— съёмка внезапно прерывается и~начинается другая;
    \item постепенное исчезновение или проявление (в монохромном кадре);
    \item вытеснение\index{Вытеснение}~— исчезновения одной съёмки, и~появления другой
        (растворение, вытеснение шторкой).
\end{itemize}

Для обнаружения границ съёмок, как правило,
сначала извлекают визуальные характеристики каждого кадра.
Затем, на~основе выделенных признаков, оценивают сходство между кадрами.
Границы съёмок определяют по смене неоднородных кадров.
В~работе \cite{Nigay:1993} описаны параметры смены кадров
и~ошибки выделения на~основе глобальных и~локальных характеристик
для~обнаружения съёмок и~классификации.

Существует два типа методов обнаружения съёмок.
\begin{enumerate}
    \item Пороговые~— попарно сравнивают подобия кадров с~заданным порогом.
    \item Статистические~— обнаруживают границы сцен на~основе характеристик кадров.
\end{enumerate}

\subsection{Извлечение ключевых кадров}

Среди кадров одной съёмки есть избыточность.
Для ее уменьшения выделяют кадры,
которые наиболее полно отражают содержание съёмки.

При~извлечении ключевых кадров\index{Ключевой кадр}\
используют различные характеристики:
\begin{itemize}
    \item цветовые гистограммы\index{Цветовая гистограмма};
    \item края;
    \item очертания;
    \item оптические потоки\index{Оптический поток}.
\end{itemize}

Способы извлечения подразделяются на~шесть категорий:
\begin{itemize}
    \item последовательное сравнение;
    \item глобальное сравнение;
    \item на~основе ссылочных кадров\index{Ссылочный кадр};
    \item на~основе кластеризации;
    \item на~основе упрощения кривых;
    \item и~на~основе объектов или событий \cite{Truong:2007}.
\end{itemize}

При~последовательном сравнении ключевой кадр сравнивают
с~другими кадрами до тех пор пока не будет найден «сильно отличный».
Для сопоставления кадров используется цветовые гистограммы \cite{Zhang:2003}.

Методы глобального сравнения используют различия между кадрами
в~съёмке и~распределяют ключевые кадры,
минимизируя предопределенную целевую функцию.

Методы на~основе ссылочных кадров генерируют систему отсчета кадров
и~затем сравнивают ключевые кадры съёмки со ссылочным.

В~работе \cite{Kazunori:2006} описано создание средней
гистограммы без канала прозрачности.
С~помощью такой гистограммы описывается цветовое распределение кадров в~съёмке.


\subsection{Сегментация сцен}

Сегментация сцен также известна как деление сюжета на~блоки.

Сцена представляет собой группу смежных съёмок.
Эти съёмки связаны между собой конкретной темой или предметом.
Сцены обладают семантикой более высокого уровня чем съёмки.

Существует три способа сегментации сцен:
\begin{itemize}
    \item деление по ключевому кадру;
    \item деление на~основе объединения визуальной и~звуковой информации;
    \item деление по фону.
\end{itemize}

При~делении сцен по ключевому кадру каждая съёмка представляется набором
ключевых кадров. Для~кадров выявляют их характеристики.
Близкие по~времени кадры с~близкими
характеристиками группируют в~сцены \cite{Truong:2003}.
Далее, используя сравнение блоков ключевых кадров,
вычисляют сходство между съёмками,
Ограничение деления по ключевому кадру заключается в~том, что кадры
не могут эффективно представить динамическое содержание съёмки.

Съёмки в~пределах сцены, как правило,
связаны динамическим развитием сюжета в~пределах сцены,
а~не~сходством ключевых кадров.


При~одновременном анализе звуковой и~визуальной информации сменой сцен считают
границу съёмки, где содержимое обоих каналов изменяется одновременно.
Для определения соответствия между этими двумя наборами сцен
используют алгоритм поиска ближайшего соседа
с~ограничением по времени \cite{Sundaram:2000}.
К~минусам подхода следует отнести сложность
определения связи между аудио сегментами и~визуальными съёмками.

Деление сцен по фону основано на~гипотезе, что съёмки,
принадлежащие к~одной сцене часто имеют один и~тот же фон.
Для восстановления фона каждого кадра используют объединение
близких по цвету пикселей в~одноцветные прямоугольные области.
Сходство съёмок определяют с помощью оценки распределения
цвета и~текстуры всех фоновых изображений в~кадре.
Для управлением процессом группировки съёмок
применяют кинематографические правила \cite{Chen:2008}.

\subsection{Сегментация звука}

Звуковая дорожка~— богатый источник информации
о~содержании для~всех жанров видео.

Как показано в~лингвистической литературе границы «высказываний»
выделяются интонационно\index{Интонация}.
На~существенные изменения темы обычно указывают:
\begin{itemize}
    \item длинные паузы;
    \item изменения тона;
    \item и~более изменением амплитуды колебаний.
\end{itemize}

Для автоматического деления речи на~темы
применяется вероятностная модель связи интонационных
и~лексических сигналов.
Сначала извлекают большое количество интонационных характеристик.
И, таким образом, получают два главных типа речевой просодии\index{Просодия}:
продолжительность и~тон.

На основе дерева принятия решений выбирают
типичную интонационную функцию.
После чего, лексическая информация извлекается
с~помощью Скрытых Моделей Маркова\index{Скрытые Модели Маркова}\
(HMM\index{HMM}) и~статистических моделей языка.

Аудио является перспективным источником информации
для~анализа лекционных видео.
Обычно такие видео длятся $60 – 90$ минут.
Сложно искать интересующий отрывок по всему такому видео \cite{Repp:2008}.
Для решения проблемы используют технологии
распознавания речи\index{Распознавание речи}.
Сначала текст извлекают из~аудио,
а~потом производят индексацию стенограммы для~поиска по ней \cite{Kumar:2011}.
Например, система распознавания речи Sphinx-4\index{Sphinx-4}\ при~поиске
по~видео достигает полноты 72\% и~средней точности 84\%.

\section{Выделение признаков}

Из~полученных частей видео выделяют признаки.
К~признакам относят:
\begin{itemize}
    \item характеристики ключевых кадров;
    \item объекты;
    \item движение в~кадре;
    \item характеристики аудио и~текста.
\end{itemize}


\begin{figuredt}
    \import{tikz/}{feature-classification}
    \fcaption{Классификация признаков для выделения при ассоциативном видео-поиске}
\end{figuredt}


\subsection{Характеристики ключевых кадров}

Выделяют цветовые, текстурные, формовые, краевые характеристики.

\subsubsection{Цветa}

Цветовые характеристики включают в~себя цветовые гистограммы,
цветовые моменты\index{Цветовой момент},
цветовые коррелограммы\index{Цветовая коррелограмма},
смесь Гауссовых моделей\index{Гауссовы модели}.
При~выделении локальной цветовой информации изображения разбивают
на~блоки $5 \times 5$ \cite{Yan:2007}.


\subsubsection{Текстуры}

Текстурными характеристиками называют визуальные особенности поверхности
некоторого объекта. Они не~зависят от~тона или насыщенности цвета объекта.
Текстурные характеристики отражают однородные явления в~изображениях.
Для выделение текстурной информации
из~видео применяют фильтры Габора\index{Фильтры Габора}\ \cite{Adcock:2004}.

\subsubsection{Контуры}

Контурные или формовые характеристики, описывают формы объектов в~изображениях.
Они могут быть извлечены из~контуров или областей объектов.


\subsubsection{Края}

На конференции TRECVid-2005 для~получения
пространственного распределение краев в~задаче поиска по видео был предложен
дескриптор гистограммы границ (EHD\index{EHD}) \cite{Hauptmann:2003}.

\subsection{Характеристики объектов}

Такие характеристики включают параметры областей изображения,
которые соответствуют объектам:
\begin{itemize}
    \item основной цвет;
    \item текстуру;
    \item размер и~т.д.
\end{itemize}

В~работе \cite{Sivic:2005} предложена система поиска лиц.
По видео-запросу с~конкретным человеком
система способна выдать ранжированный список съёмок с~этим человеком.
Текстовая индексация и~поиск приводят к~расширению семантики запроса
и~делают возможным использования Glimpse-метода\index{Glimpse}\ (agrep\index{Agrep}),
для~поиска нечеткого соответствия \cite{Li:2002}.


\subsection{Характеристики движения}

Характеристики движения ближе к~смысловым понятиям,
чем характеристики ключевых статических кадров и~объектов.
Движение в~видео может быть вызвано движением камеры
и~движением предметов в кадре.

Движения камеры такие как «приближение или удаление»,
«панорамирование влево или вправо» и~«смещение вверх или вниз»
используются для~индексации видео.
Движения объектов на~данный момент являются предметном исследований.

\subsection{Звуковые характеристики}

Преимущество аудио-подходов состоит в~том,
что они обычно требуют меньше вычислительных ресурсов, чем визуальные методы.
Кроме того, аудио-записи могут быть очень короткими.

Многие звуковые характеристики выбраны на~основе человеческого восприятие звука.
Характеристики аудио можно разделить на~три уровня \cite{Chen:2008}:
\begin{itemize}
    \item низкоуровневая акустика, такая как средняя частота для~кадра,
    \item средний уровень, как признак объекта, например звук скачущего мяча,
    \item высокоуровневые, такие как речь и фоновая музыка,
        играющая в~определенных типах видео.
\end{itemize}


В работе \cite{Seyerlehner:2010} используют блочные характеристики аудио.
Аудио-поток при этом разделяется на отрезки в 2048 отсчетов.
Для выделения таких характеристик применяют функцию Ханна\index{Функция Ханна}\
и логарифмическую шкалу.



\section{Представление видео}


В~работе \cite{Haase:95} была сформулировала проблема
машинного представления видео.
В~работе \cite{Chih-Wen:2007} разработаны многослойные,
графические аннотации видео~— мультимедийные
потоки\index{Мультимедийный поток}.
Они представляют собой визуальный язык
как~способ представления видеоданных. Особое внимание уделено
проблеме создания глобального архива видео,
допускающего повторное использование.
Нисходящие поисковые системы используют высокоуровневые знания определенной
предметной области, чтобы генерировать надлежащие представления.

Но как было сказано выше, это не самый удобный подход.
Представление, управляемое данными~— стандартный способ извлечь
низкоуровневые характеристики и~получить соответствующие представления
без любых предварительных знаний о~предметной области.

Представления, на основе данных могут быть сведены к~двум основным классам.
\begin{enumerate}
    \item Сигнальные признаки, которые характеризуют низкоуровневое
        аудиовизуальное содержание. К ним можно отнести
        цветовые гистограммы, формы, текстуры,
    \item Описательное представление с~помощью текста, атрибутов или ключевых слов.
        Авторы работы \cite{Xiangang:2011} предлагают для~описания
        видео использовать послойные графовые клики ключевых кадров
        (SKCs\index{SKC}), которые более компактны и~информативны,
        чем последовательность изображений или ключевые кадры.
\end{enumerate}



\begin{figuredt}
    \import{tikz/}{video-representation}
    \fcaption{Способы представления видео}
\end{figuredt}


\section{Анализ видео}

Интеллектуальный анализ данных в~больших базах видео стал доступен недавно.
Задачи анализа видеоинформации можно сформулировать как выявление:
\begin{itemize}
    \item структурных закономерностей видео;
    \item закономерностей поведения движущихся объектов;
    \item характеристик сцены;
    \item шаблонов событий и~их связей;
    \item и~других смысловых атрибутов в~видео.
\end{itemize}

В работах применяют извлечение объектов~— группировку различных экземпляров
того же объекта, который появляется в~различных частях видео.

Для классификации пространственных характеристик кадров применяют
метод поиска ближайших соседей\index{Поиск ближайшего соседа}\index{NNS}\
\cite{Anjulan:2009}.
%При поиске подобных объектов выделяют стабильные связи,
%которые объединены в~значимые объектные кластеры.

Обнаружение специальных шаблонов применяется к~действиям и событиям,
для~которых есть априорные модели, такие как действия человека,
спортивные мероприятия, дорожные ситуации
или образцы преступлений \cite{Quack:2006}.

Поиск моделей~— автоматическое извлечение неизвестных закономерностей в~видео.
Для поиска моделей используют экспертные системы
с~безнадзорным\index{Обучение!Без учителя}\
или полуконтролируемым обучением\index{Обучение!Полуконтролируемое}.

Поиск неизвестных моделей полезен
для~изучения новых данных в~наборе видео.
Неизвестные образцы обычно находят благодаря
кластеризации различных векторов характеристик.

Для выявления закономерностей поведения движущихся объектов
используют n-граммы\index{N-грамма}\ и~суффиксные
деревья\index{Суффиксное дерево}.
При~этом анализируют последовательности событий
по~многократным временным масштабам.


\section{Классификация видео}

Задача классификации состоит в~том,
чтобы отнести видео к~предопределенной категории.
Для этого используют характеристики видео или
результаты интеллектуального анализа данных.

Классификация видео~— хороший способ увеличить
эффективность видео-поиска.
Семантический разрыв между низкоуровневыми данными
и~интерпретацией наблюдателя, делает ассоциативную классификацию
очень трудной задачей.

Смысловая классификация видео может быть выполнена
на~трех уровнях \cite{Tamizharasan:2013}:
\begin{itemize}
    \item жанры
    \begin{itemize}
        \item например, «фильмы», «новости»,
                «спортивные соревнования», «мультфильмы», «реклама» и~т.д.
    \end{itemize}
    \item события видео;
    \item и~объекты в~видео.
\end{itemize}


\begin{figuredt}
    \import{tikz/}{video-levels}
    \fcaption{Уровни классификации видео}
\end{figuredt}



\subsection{Жанры}

Жанровая классификация разделяет видео на~подмножество соответствующее жанру
и~несоответствующее \cite{Wu:2012}.

В~работе \cite{Jiang:2007} предложена классификация большого числа
видео только по заголовку видео.
Для этого использован поэтапный метод опорных
векторов\index{Метод опорных векторов}.

Видео классифицируют также на~основе статистических моделей различных жанров.
Для этого анализируют структурные свойства:
статистику цвета, съёмки, движение камеры и~объектов.
Свойства используются, чтобы получить более абстрактные атрибуты стиля.
К абстрактным атрибутам стиля можно отнести:
панорамирование камеры и~изменение масштаба, речь и~музыку.
Строят отображение этих атрибутов на~жанры видео.

В~работе \cite{Ionescu:2012} для классификации жанров используется
комбинация из четырех дескрипторов:
\begin{itemize}
    \item блоковый аудио дескриптор
    \begin{itemize}
        \item захватывает локальную временную информацию;
    \end{itemize}
    \item дескриптор визуальной временной структуры
    \begin{itemize}
        \item использует информацию о смене съёмок,
        \item оценивает количество съёмок
            за определенный интервал времени («ритм» видео),
        \item описывает об «активные» и «не актиные» смены съемок;
    \end{itemize}
    \item дескриптор цвета
    \begin{itemize}
        \item использует статистику распределения цвета,
        элементарных оттенков, цветовых свойств, и отношений между цветами;
    \end{itemize}
    \item статистика фигур контуров.
\end{itemize}

Были проведены эксперименты на видеоматериалах общей продолжительностью
91 часе видео. Классификация проводилась на семи жанрах видео:
мультфильмы, реклама, документальные фильмы, художественные фильмы,
музыкальные клипы, спортивные соревнования и новости.
Комплексный дескриптор позволил авторам достичь
точности  87\% -100\% и полноты 77\% -100\%.



\subsection{События}

Событие может быть определено как любое явление в~видео,
которое
\begin{itemize}
    \item может быть воспринято зрителем;
    \item играет роль для~представления содержимого.
\end{itemize}
Каждое видео может состоять из~многих событий,
и~каждое событие может состоять из~многих подсобытий.
Таким образом складывается иерархическая модель \cite{Chang:2002}.

\subsection{Объекты}

Объектная классификация является самым низкоуровневым типом классификации.
Съёмки классифицируют тоже на~основе объектов.
Объекты в~съёмках представлены с~помощью параметров цвета, текстуры и~траектории.
В~работе \cite{Hong:2005} для~кластеризации связанных съёмок
используется нейронная сеть.
Каждый кластер отображен на~одну из~12 категорий.
Объекты разделяются по положению в кадре и характеру движения.


\section{Аннотирование видео}

Процесс присваивания переопределенных смысловых понятий фрагментам видео
называют аннотированием. Примеры смысловых понятий : человек,
автомобиль, небо и~гуляющие люди.

Аннотирование видео подобно классификации, за исключением двух различий.
\begin{enumerate}
    \item Для классификаций важны жанры, а~для~аннотирования понятия.
        Жанры и~понятия имеют различную природу, несмотря на~то,
        что некоторые методы могут быть использованы в~обоих задачах.
    \item Классификация видео применяется к~полным видео,
        в~то время как аннотируют обычно фрагменты \cite{Yang:2007}.
\end{enumerate}

Аннотирование, основанное на~обучении, необходимо для~анализа
и~понимания видео. Было предложено много различных способов
автоматизации процесса.

Например, в~работе \cite{Zhang:2012} было
разработано «быстрое полуконтролируемое\index{Обучение!Полуконтролируемое}\
графовое обучение на~нескольких экземплярах»
(Fast Graphbased Semi-Supervised Multiple Instance Learning — FGSSMIL\index{FGSSMIL}).
Алгоритм работает в~рамках общей платформы для~разных типов видео одновременно
(спортивные передачи, новости, художественные фильмы).
Для обучения модели используется небольшое число видео, размеченных вручную,
и~значительный объем не~размеченного материала.

В~работе \cite{Weal:2012} предлагается создавать
частичную ручную аннотацию видео
как часть практической профессиональной подготовки.
Авторы рассматривают лабораторные занятия студентов-медиков.
Во~время занятия идет запись видео.
Кроме того одновременно происходит запись изменения состояния
тренировочного манекена (виртуального пациента).
Таким образом, к~записанному видео добавляется семантическая разметка
на~основе показаний датчиков манекена.
После происходит разбор занятия и~анализ допущенных ошибок,
В результате к~видео добавляется разметка, созданная самими студентами.


\section{Обработка запроса}

После построения поискового индекса может быть выполнен ассоциативный поиск.
Поисковая выдача оптимизируются на~основе связи между запросами.

\subsection{Типы запросов}


Существует две категории запросов:
семантические и~не семантические.


\begin{figuredt}
    \import{tikz/}{video-query-types}
    \fcaption{Типы запросов при ассоциативном видео-поиске}
\end{figuredt}


\subsubsection{Семантические запросы}


К~семантическим запросам относят наборы ключевых слов
и~поисковые фразы.
Ключевые слова~— наиболее очевидный и~простой вид запроса.
При~таких запросах частично учитывается семантика видео.
Поисковые фразы или запросы на~естественном языке~—
самый естественный и~удобный способ взаимодействия
человека с~поисковой системой. Для выбора и~ранжирования видео
используется смысловая близость слов \cite{Aytar:2008}.

\subsubsection{Не семантические запросы}

Не~семантические запросы используются для~поиска по образцу,
эскизам, объектам и~т.д..
Запросом может быть изображение или видео.

\paragraph{Поиск по образцу}

При~таком поиске из~запроса выделяют низкоуровневые характеристики
и~сравнивают их с~данными в~базе с~помощью меры сходства.

\paragraph{Поиск по эскизу}

Пользователи могут изобразить нужное видео с~помощью эскиза.
Далее для~эскиза применяется поиск по образцу.

\paragraph{Поиск объекта}

В~качестве запроса выступает изображение объекта.
Система находит и~возвращает все вхождения объекта
в~материалах из~базы \cite{Sivic:2006}.
В~отличие от~предыдущих видов запросов,
в~данном случае, привязка происходит не~к видео,
а~именно по изображенному объекту.


\subsection{Оценка сходства}


Критерии близости видео является важным фактором при~поиске.
Выделяют несколько способов сравнения видео:
\begin{itemize}
    \item сравнение характеристик;
    \item сравнение текста;
    \item сравнение онтологий.
\end{itemize}

Применяют также комбинации методов.
Выбор конкретного метода зависит от~типа запроса.

\subsection{Сравнение характеристик}

При~сравнение характеристик видео оценивают среднее расстояние между
особенностями соответствующих кадров \cite{Browne:2005}.

\subsection{Сравнение текста}

Для~сравнения запроса и описания видео
применяют текстовое сопоставление.
Описание и~запрос нормализуют,
а~затем вычисляют их смысловое сходство,
используя пространственные векторные модели \cite{Snoek:2007}.

\subsection{Сравнение онтологий}

При~сравнении онтологии оценивают смысловое сходство
отношений между ключевыми словами запроса
и~описанием аннотированного видео \cite{Aytar:2008}.

Для усиления влияния смысловых понятий
автоматически подбирают комбинации методов.
Для этого исследуют различные стратегии на~учебном наборе видео.

\subsection{Оценка релевантности}

Видео из~поисковой выдачи оцениваются или пользователем или автоматически.
Эту оценку используют для~уточнения дальнейших поисков.
Обратная связь релевантности устраняет разрыв между смысловым понятием
адекватности поискового ответа и~низкоуровневым представлением видео.

Явная обратная связь\index{Обратная связь!Явная}\ предлагает пользователю выбрать
релевантные видеоролики из~ранее полученных ответов.
На основе мнений пользователей в~системы меняют коэффициенты
мер подобия \cite{Chen:2008:a}.

Неявная обратная связь\index{Обратная связь!Неявная}\ уточняет результаты поиска
на~основе кликов и~переходов пользователя.

Псевдообратная связь\index{Обратная связь!Псевдообратная}\
выделяет положительные и~отрицательные выборки
из~предыдущих результатов поиска без участия человека.

Рассматривая текстовую и~визуальную информацию
с~вероятностной точки зрения, визуальное ранжирование
можно сформулировать как задачу байесовской оптимизации\index{Байесовская оптимизация}.
Такое прием называют байесовским визуальным ранжированием.


\section{Заключение}

Многие вопросы остаются открытыми и~требуют дальнейшего исследования,
особенно в~следующих областях.

Большинство современных подходов индексации видео
сильно зависят от~предварительных знаний о предметной области.
Это ограничивает их~расширяемость для~новых областей.
Устранение зависимости от~предварительных знаний~—
важная задача будущих исследований.

Индексация и~поиск видео в~среде «облачных» вычислений сформировали
новое направление исследований видео-поиска.
Важной особенностью «облачных» вычислений является то,
что искомые видео и~сама база данных меняются динамически.

Современные подходы к~смысловому поиску видео,
как правило, используют набор текстов для~описания визуального содержания видео.
В~этой области пока осталось много неразрешенных вопросов.
Например, отдельной темой для~исследования может быть
эмоциональная семантика видео \cite{Tamizharasan:2013}.
Эмоциональная семантика описывает
человеческие психологические ощущения,
такие как радость, гнев, страх, печаль, и~пр.

Эмоциональный видео-поиск~— поиск материалов,
которые вызывают конкретные чувства у~зрителя.
Для имитации человеческого восприятия могут быть использованы
новые подходы к~видео-поиску.

Темой для~дальнейшего изучения является
мультимедийный человеко-машинный интерфейс,
в~частности:
\begin{itemize}
    \item расположение мультимедийной информации;
    \item удобство интерфейса для~решения задач пользователя;
    \item пригодность интерфейса для~оценки и~обратной связи пользователей;
    \item и~способность интерфейса адаптироваться к~привычкам запроса пользователей
        и~отражать их индивидуальность.
\end{itemize}

Организация и~визуализация результатов поиска~— также
интересная тема исследования. На данный момент проблема
сочетания множественных информационных моделей
на~различных уровнях абстракции остается неразрешенной.

Эффективное использование информации о~движении имеет большое значение
для~поиска видео. Важными задачами направления являются:
\begin{itemize}
    \item способность различать движения фона и~переднего плана;
    \item обнаружение движущиеся объектов и~определять события в~кадре;
    \item объединение статических характеристик и~характеристик движения;
    \item построение индекса движения.
\end{itemize}

Интересными вопросами для~исследования остаются:
\begin{itemize}
    \item быстрый видео-поиск с~помощью иерархических индексов;
    \item адаптивное обновление иерархической индексной модели;
    \item обработка временных характеристик видео во~время создания и~обновления индекса;
    \item динамические меры сходства видео на~основе выбора статистических функций.
\end{itemize}


