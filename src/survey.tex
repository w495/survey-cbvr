

\subimport{survey/}{abstract-ru}

\subimport{survey/}{abstract-en}

\pagebreak

\subimport{survey/}{intro}

\subimport{survey/}{segmentation}

\subimport{survey/}{features}


\section{Представление видео}


В~работе \cite{Haase:95} была сформулировала проблема
машинного представления видео.
В~работе \cite{Chih-Wen:2007} разработаны многослойные,
графические аннотации видео~— мультимедийные
потоки\index{Мультимедийный поток}.
Они представляют собой визуальный язык
как~способ представления видеоданных. Особое внимание уделено
проблеме создания глобального архива видео,
допускающего повторное использование.
Нисходящие поисковые системы используют высокоуровневые знания определенной
предметной области, чтобы генерировать надлежащие представления.

Но как было сказано выше, это не самый удобный подход.
Представление, управляемое данными~— стандартный способ извлечь
низкоуровневые характеристики и~получить соответствующие представления
без любых предварительных знаний о~предметной области.

Представления, на основе данных могут быть сведены к~двум основным классам.
\begin{enumerate}
    \item Сигнальные признаки, которые характеризуют низкоуровневое
        аудиовизуальное содержание. К ним можно отнести
        цветовые гистограммы, формы, текстуры,
    \item Описательное представление с~помощью текста, атрибутов или ключевых слов.
        Авторы работы \cite{Xiangang:2011} предлагают для~описания
        видео использовать послойные графовые клики ключевых кадров
        (SKCs\index{SKC}), которые более компактны и~информативны,
        чем последовательность изображений или ключевые кадры.
\end{enumerate}



\begin{figuredt}
    \import{tikz/}{video-representation}
    \fcaption{Способы представления видео}
\end{figuredt}


\section{Анализ видео}

Интеллектуальный анализ данных в~больших базах видео стал доступен недавно.
Задачи анализа видеоинформации можно сформулировать как выявление:
\begin{itemize}
    \item структурных закономерностей видео;
    \item закономерностей поведения движущихся объектов;
    \item характеристик сцены;
    \item шаблонов событий и~их связей;
    \item и~других смысловых атрибутов в~видео.
\end{itemize}

В работах применяют извлечение объектов~— группировку различных экземпляров
того же объекта, который появляется в~различных частях видео.

Для классификации пространственных характеристик кадров применяют
метод поиска ближайших соседей\index{Поиск ближайшего соседа}\index{NNS}\
\cite{Anjulan:2009}.
%При поиске подобных объектов выделяют стабильные связи,
%которые объединены в~значимые объектные кластеры.

Обнаружение специальных шаблонов применяется к~действиям и событиям,
для~которых есть априорные модели, такие как действия человека,
спортивные мероприятия, дорожные ситуации
или образцы преступлений \cite{Quack:2006}.

Поиск моделей~— автоматическое извлечение неизвестных закономерностей в~видео.
Для поиска моделей используют экспертные системы
с~безнадзорным\index{Обучение!Без учителя}\
или полуконтролируемым обучением\index{Обучение!Полуконтролируемое}.

Поиск неизвестных моделей полезен
для~изучения новых данных в~наборе видео.
Неизвестные образцы обычно находят благодаря
кластеризации различных векторов характеристик.

Для выявления закономерностей поведения движущихся объектов
используют n-граммы\index{N-грамма}\ и~суффиксные
деревья\index{Суффиксное дерево}.
При~этом анализируют последовательности событий
по~многократным временным масштабам.


\section{Классификация видео}

Задача классификации состоит в~том,
чтобы отнести видео к~предопределенной категории.
Для этого используют характеристики видео или
результаты интеллектуального анализа данных.

Классификация видео~— хороший способ увеличить
эффективность видео-поиска.
Семантический разрыв между низкоуровневыми данными
и~интерпретацией наблюдателя, делает ассоциативную классификацию
очень трудной задачей.

Смысловая классификация видео может быть выполнена
на~трех уровнях \cite{Tamizharasan:2013}:
\begin{itemize}
    \item жанры
    \begin{itemize}
        \item например, «фильмы», «новости»,
                «спортивные соревнования», «мультфильмы», «реклама» и~т.д.
    \end{itemize}
    \item события видео;
    \item и~объекты в~видео.
\end{itemize}


\begin{figuredt}
    \import{tikz/}{video-levels}
    \fcaption{Уровни классификации видео}
\end{figuredt}



\subsection{Жанры}

Жанровая классификация разделяет видео на~подмножество соответствующее жанру
и~несоответствующее \cite{Wu:2012}.

В~работе \cite{Jiang:2007} предложена классификация большого числа
видео только по заголовку видео.
Для этого использован поэтапный метод опорных
векторов\index{Метод опорных векторов}.

Видео классифицируют также на~основе статистических моделей различных жанров.
Для этого анализируют структурные свойства:
статистику цвета, съёмки, движение камеры и~объектов.
Свойства используются, чтобы получить более абстрактные атрибуты стиля.
К абстрактным атрибутам стиля можно отнести:
панорамирование камеры и~изменение масштаба, речь и~музыку.
Строят отображение этих атрибутов на~жанры видео.

В~работе \cite{Ionescu:2012} для классификации жанров используется
комбинация из четырех дескрипторов:
\begin{itemize}
    \item блоковый аудио дескриптор
    \begin{itemize}
        \item захватывает локальную временную информацию;
    \end{itemize}
    \item дескриптор визуальной временной структуры
    \begin{itemize}
        \item использует информацию о смене съёмок,
        \item оценивает количество съёмок
            за определенный интервал времени («ритм» видео),
        \item описывает об «активные» и «не актиные» смены съемок;
    \end{itemize}
    \item дескриптор цвета
    \begin{itemize}
        \item использует статистику распределения цвета,
        элементарных оттенков, цветовых свойств, и отношений между цветами;
    \end{itemize}
    \item статистика фигур контуров.
\end{itemize}

Были проведены эксперименты на видеоматериалах общей продолжительностью
91 часе видео. Классификация проводилась на семи жанрах видео:
мультфильмы, реклама, документальные фильмы, художественные фильмы,
музыкальные клипы, спортивные соревнования и новости.
Комплексный дескриптор позволил авторам достичь
точности  87\% -100\% и полноты 77\% -100\%.



\subsection{События}

Событие может быть определено как любое явление в~видео,
которое
\begin{itemize}
    \item может быть воспринято зрителем;
    \item играет роль для~представления содержимого.
\end{itemize}
Каждое видео может состоять из~многих событий,
и~каждое событие может состоять из~многих подсобытий.
Таким образом складывается иерархическая модель \cite{Chang:2002}.

\subsection{Объекты}

Объектная классификация является самым низкоуровневым типом классификации.
Съёмки классифицируют тоже на~основе объектов.
Объекты в~съёмках представлены с~помощью параметров цвета, текстуры и~траектории.
В~работе \cite{Hong:2005} для~кластеризации связанных съёмок
используется нейронная сеть.
Каждый кластер отображен на~одну из~12 категорий.
Объекты разделяются по положению в кадре и характеру движения.


\section{Аннотирование видео}

Процесс присваивания переопределенных смысловых понятий фрагментам видео
называют аннотированием. Примеры смысловых понятий : человек,
автомобиль, небо и~гуляющие люди.

Аннотирование видео подобно классификации, за исключением двух различий.
\begin{enumerate}
    \item Для классификаций важны жанры, а~для~аннотирования понятия.
        Жанры и~понятия имеют различную природу, несмотря на~то,
        что некоторые методы могут быть использованы в~обоих задачах.
    \item Классификация видео применяется к~полным видео,
        в~то время как аннотируют обычно фрагменты \cite{Yang:2007}.
\end{enumerate}

Аннотирование, основанное на~обучении, необходимо для~анализа
и~понимания видео. Было предложено много различных способов
автоматизации процесса.

Например, в~работе \cite{Zhang:2012} было
разработано «быстрое полуконтролируемое\index{Обучение!Полуконтролируемое}\
графовое обучение на~нескольких экземплярах»
(Fast Graphbased Semi-Supervised Multiple Instance Learning — FGSSMIL\index{FGSSMIL}).
Алгоритм работает в~рамках общей платформы для~разных типов видео одновременно
(спортивные передачи, новости, художественные фильмы).
Для обучения модели используется небольшое число видео, размеченных вручную,
и~значительный объем не~размеченного материала.

В~работе \cite{Weal:2012} предлагается создавать
частичную ручную аннотацию видео
как часть практической профессиональной подготовки.
Авторы рассматривают лабораторные занятия студентов-медиков.
Во~время занятия идет запись видео.
Кроме того одновременно происходит запись изменения состояния
тренировочного манекена (виртуального пациента).
Таким образом, к~записанному видео добавляется семантическая разметка
на~основе показаний датчиков манекена.
После происходит разбор занятия и~анализ допущенных ошибок,
В результате к~видео добавляется разметка, созданная самими студентами.


\section{Обработка запроса}

После построения поискового индекса может быть выполнен ассоциативный поиск.
Поисковая выдача оптимизируются на~основе связи между запросами.

\subsection{Типы запросов}


Существует две категории запросов:
семантические и~не семантические.


\begin{figuredt}
    \import{tikz/}{video-query-types}
    \fcaption{Типы запросов при ассоциативном видео-поиске}
\end{figuredt}


\subsubsection{Семантические запросы}


К~семантическим запросам относят наборы ключевых слов
и~поисковые фразы.
Ключевые слова~— наиболее очевидный и~простой вид запроса.
При~таких запросах частично учитывается семантика видео.
Поисковые фразы или запросы на~естественном языке~—
самый естественный и~удобный способ взаимодействия
человека с~поисковой системой. Для выбора и~ранжирования видео
используется смысловая близость слов \cite{Aytar:2008}.

\subsubsection{Не семантические запросы}

Не~семантические запросы используются для~поиска по образцу,
эскизам, объектам и~т.д..
Запросом может быть изображение или видео.

\paragraph{Поиск по образцу}

При~таком поиске из~запроса выделяют низкоуровневые характеристики
и~сравнивают их с~данными в~базе с~помощью меры сходства.

\paragraph{Поиск по эскизу}

Пользователи могут изобразить нужное видео с~помощью эскиза.
Далее для~эскиза применяется поиск по образцу.

\paragraph{Поиск объекта}

В~качестве запроса выступает изображение объекта.
Система находит и~возвращает все вхождения объекта
в~материалах из~базы \cite{Sivic:2006}.
В~отличие от~предыдущих видов запросов,
в~данном случае, привязка происходит не~к видео,
а~именно по изображенному объекту.


\subsection{Оценка сходства}


Критерии близости видео является важным фактором при~поиске.
Выделяют несколько способов сравнения видео:
\begin{itemize}
    \item сравнение характеристик;
    \item сравнение текста;
    \item сравнение онтологий.
\end{itemize}

Применяют также комбинации методов.
Выбор конкретного метода зависит от~типа запроса.

\subsection{Сравнение характеристик}

При~сравнение характеристик видео оценивают среднее расстояние между
особенностями соответствующих кадров \cite{Browne:2005}.

\subsection{Сравнение текста}

Для~сравнения запроса и описания видео
применяют текстовое сопоставление.
Описание и~запрос нормализуют,
а~затем вычисляют их смысловое сходство,
используя пространственные векторные модели \cite{Snoek:2007}.

\subsection{Сравнение онтологий}

При~сравнении онтологии оценивают смысловое сходство
отношений между ключевыми словами запроса
и~описанием аннотированного видео \cite{Aytar:2008}.

Для усиления влияния смысловых понятий
автоматически подбирают комбинации методов.
Для этого исследуют различные стратегии на~учебном наборе видео.

\subsection{Оценка релевантности}

Видео из~поисковой выдачи оцениваются или пользователем или автоматически.
Эту оценку используют для~уточнения дальнейших поисков.
Обратная связь релевантности устраняет разрыв между смысловым понятием
адекватности поискового ответа и~низкоуровневым представлением видео.

Явная обратная связь\index{Обратная связь!Явная}\ предлагает пользователю выбрать
релевантные видеоролики из~ранее полученных ответов.
На основе мнений пользователей в~системы меняют коэффициенты
мер подобия \cite{Chen:2008:a}.

Неявная обратная связь\index{Обратная связь!Неявная}\ уточняет результаты поиска
на~основе кликов и~переходов пользователя.

Псевдообратная связь\index{Обратная связь!Псевдообратная}\
выделяет положительные и~отрицательные выборки
из~предыдущих результатов поиска без участия человека.

Рассматривая текстовую и~визуальную информацию
с~вероятностной точки зрения, визуальное ранжирование
можно сформулировать как задачу байесовской оптимизации\index{Байесовская оптимизация}.
Такое прием называют байесовским визуальным ранжированием.


\section{Заключение}

Многие вопросы остаются открытыми и~требуют дальнейшего исследования,
особенно в~следующих областях.

Большинство современных подходов индексации видео
сильно зависят от~предварительных знаний о предметной области.
Это ограничивает их~расширяемость для~новых областей.
Устранение зависимости от~предварительных знаний~—
важная задача будущих исследований.

Индексация и~поиск видео в~среде «облачных» вычислений сформировали
новое направление исследований видео-поиска.
Важной особенностью «облачных» вычислений является то,
что искомые видео и~сама база данных меняются динамически.

Современные подходы к~смысловому поиску видео,
как правило, используют набор текстов для~описания визуального содержания видео.
В~этой области пока осталось много неразрешенных вопросов.
Например, отдельной темой для~исследования может быть
эмоциональная семантика видео \cite{Tamizharasan:2013}.
Эмоциональная семантика описывает
человеческие психологические ощущения,
такие как радость, гнев, страх, печаль, и~пр.

Эмоциональный видео-поиск~— поиск материалов,
которые вызывают конкретные чувства у~зрителя.
Для имитации человеческого восприятия могут быть использованы
новые подходы к~видео-поиску.

Темой для~дальнейшего изучения является
мультимедийный человеко-машинный интерфейс,
в~частности:
\begin{itemize}
    \item расположение мультимедийной информации;
    \item удобство интерфейса для~решения задач пользователя;
    \item пригодность интерфейса для~оценки и~обратной связи пользователей;
    \item и~способность интерфейса адаптироваться к~привычкам запроса пользователей
        и~отражать их индивидуальность.
\end{itemize}

Организация и~визуализация результатов поиска~— также
интересная тема исследования. На данный момент проблема
сочетания множественных информационных моделей
на~различных уровнях абстракции остается неразрешенной.

Эффективное использование информации о~движении имеет большое значение
для~поиска видео. Важными задачами направления являются:
\begin{itemize}
    \item способность различать движения фона и~переднего плана;
    \item обнаружение движущиеся объектов и~определять события в~кадре;
    \item объединение статических характеристик и~характеристик движения;
    \item построение индекса движения.
\end{itemize}

Интересными вопросами для~исследования остаются:
\begin{itemize}
    \item быстрый видео-поиск с~помощью иерархических индексов;
    \item адаптивное обновление иерархической индексной модели;
    \item обработка временных характеристик видео во~время создания и~обновления индекса;
    \item динамические меры сходства видео на~основе выбора статистических функций.
\end{itemize}


